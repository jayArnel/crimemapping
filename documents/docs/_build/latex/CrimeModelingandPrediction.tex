% Generated by Sphinx.
\def\sphinxdocclass{report}
\documentclass[letterpaper,10pt,english]{sphinxmanual}
\usepackage{iftex}

\ifPDFTeX
  \usepackage[utf8]{inputenc}
\fi
\ifdefined\DeclareUnicodeCharacter
  \DeclareUnicodeCharacter{00A0}{\nobreakspace}
\fi
\usepackage{cmap}
\usepackage[T1]{fontenc}
\usepackage{amsmath,amssymb,amstext}
\usepackage{babel}
\usepackage{times}
\usepackage[Bjarne]{fncychap}
\usepackage{longtable}
\usepackage{sphinx}
\usepackage{multirow}
\usepackage{eqparbox}


\addto\captionsenglish{\renewcommand{\figurename}{Fig.\@ }}
\addto\captionsenglish{\renewcommand{\tablename}{Table }}
\SetupFloatingEnvironment{literal-block}{name=Listing }

\addto\extrasenglish{\def\pageautorefname{page}}

\setcounter{tocdepth}{1}


\title{Crime Modeling and Prediction Documentation}
\date{Jun 07, 2016}
\release{0.0.1}
\author{Jay Arnel D. Bilocura}
\newcommand{\sphinxlogo}{}
\renewcommand{\releasename}{Release}
\makeindex

\makeatletter
\def\PYG@reset{\let\PYG@it=\relax \let\PYG@bf=\relax%
    \let\PYG@ul=\relax \let\PYG@tc=\relax%
    \let\PYG@bc=\relax \let\PYG@ff=\relax}
\def\PYG@tok#1{\csname PYG@tok@#1\endcsname}
\def\PYG@toks#1+{\ifx\relax#1\empty\else%
    \PYG@tok{#1}\expandafter\PYG@toks\fi}
\def\PYG@do#1{\PYG@bc{\PYG@tc{\PYG@ul{%
    \PYG@it{\PYG@bf{\PYG@ff{#1}}}}}}}
\def\PYG#1#2{\PYG@reset\PYG@toks#1+\relax+\PYG@do{#2}}

\expandafter\def\csname PYG@tok@gd\endcsname{\def\PYG@tc##1{\textcolor[rgb]{0.63,0.00,0.00}{##1}}}
\expandafter\def\csname PYG@tok@gu\endcsname{\let\PYG@bf=\textbf\def\PYG@tc##1{\textcolor[rgb]{0.50,0.00,0.50}{##1}}}
\expandafter\def\csname PYG@tok@gt\endcsname{\def\PYG@tc##1{\textcolor[rgb]{0.00,0.27,0.87}{##1}}}
\expandafter\def\csname PYG@tok@gs\endcsname{\let\PYG@bf=\textbf}
\expandafter\def\csname PYG@tok@gr\endcsname{\def\PYG@tc##1{\textcolor[rgb]{1.00,0.00,0.00}{##1}}}
\expandafter\def\csname PYG@tok@cm\endcsname{\let\PYG@it=\textit\def\PYG@tc##1{\textcolor[rgb]{0.25,0.50,0.56}{##1}}}
\expandafter\def\csname PYG@tok@vg\endcsname{\def\PYG@tc##1{\textcolor[rgb]{0.73,0.38,0.84}{##1}}}
\expandafter\def\csname PYG@tok@vi\endcsname{\def\PYG@tc##1{\textcolor[rgb]{0.73,0.38,0.84}{##1}}}
\expandafter\def\csname PYG@tok@mh\endcsname{\def\PYG@tc##1{\textcolor[rgb]{0.13,0.50,0.31}{##1}}}
\expandafter\def\csname PYG@tok@cs\endcsname{\def\PYG@tc##1{\textcolor[rgb]{0.25,0.50,0.56}{##1}}\def\PYG@bc##1{\setlength{\fboxsep}{0pt}\colorbox[rgb]{1.00,0.94,0.94}{\strut ##1}}}
\expandafter\def\csname PYG@tok@ge\endcsname{\let\PYG@it=\textit}
\expandafter\def\csname PYG@tok@vc\endcsname{\def\PYG@tc##1{\textcolor[rgb]{0.73,0.38,0.84}{##1}}}
\expandafter\def\csname PYG@tok@il\endcsname{\def\PYG@tc##1{\textcolor[rgb]{0.13,0.50,0.31}{##1}}}
\expandafter\def\csname PYG@tok@go\endcsname{\def\PYG@tc##1{\textcolor[rgb]{0.20,0.20,0.20}{##1}}}
\expandafter\def\csname PYG@tok@cp\endcsname{\def\PYG@tc##1{\textcolor[rgb]{0.00,0.44,0.13}{##1}}}
\expandafter\def\csname PYG@tok@gi\endcsname{\def\PYG@tc##1{\textcolor[rgb]{0.00,0.63,0.00}{##1}}}
\expandafter\def\csname PYG@tok@gh\endcsname{\let\PYG@bf=\textbf\def\PYG@tc##1{\textcolor[rgb]{0.00,0.00,0.50}{##1}}}
\expandafter\def\csname PYG@tok@ni\endcsname{\let\PYG@bf=\textbf\def\PYG@tc##1{\textcolor[rgb]{0.84,0.33,0.22}{##1}}}
\expandafter\def\csname PYG@tok@nl\endcsname{\let\PYG@bf=\textbf\def\PYG@tc##1{\textcolor[rgb]{0.00,0.13,0.44}{##1}}}
\expandafter\def\csname PYG@tok@nn\endcsname{\let\PYG@bf=\textbf\def\PYG@tc##1{\textcolor[rgb]{0.05,0.52,0.71}{##1}}}
\expandafter\def\csname PYG@tok@no\endcsname{\def\PYG@tc##1{\textcolor[rgb]{0.38,0.68,0.84}{##1}}}
\expandafter\def\csname PYG@tok@na\endcsname{\def\PYG@tc##1{\textcolor[rgb]{0.25,0.44,0.63}{##1}}}
\expandafter\def\csname PYG@tok@nb\endcsname{\def\PYG@tc##1{\textcolor[rgb]{0.00,0.44,0.13}{##1}}}
\expandafter\def\csname PYG@tok@nc\endcsname{\let\PYG@bf=\textbf\def\PYG@tc##1{\textcolor[rgb]{0.05,0.52,0.71}{##1}}}
\expandafter\def\csname PYG@tok@nd\endcsname{\let\PYG@bf=\textbf\def\PYG@tc##1{\textcolor[rgb]{0.33,0.33,0.33}{##1}}}
\expandafter\def\csname PYG@tok@ne\endcsname{\def\PYG@tc##1{\textcolor[rgb]{0.00,0.44,0.13}{##1}}}
\expandafter\def\csname PYG@tok@nf\endcsname{\def\PYG@tc##1{\textcolor[rgb]{0.02,0.16,0.49}{##1}}}
\expandafter\def\csname PYG@tok@si\endcsname{\let\PYG@it=\textit\def\PYG@tc##1{\textcolor[rgb]{0.44,0.63,0.82}{##1}}}
\expandafter\def\csname PYG@tok@s2\endcsname{\def\PYG@tc##1{\textcolor[rgb]{0.25,0.44,0.63}{##1}}}
\expandafter\def\csname PYG@tok@nt\endcsname{\let\PYG@bf=\textbf\def\PYG@tc##1{\textcolor[rgb]{0.02,0.16,0.45}{##1}}}
\expandafter\def\csname PYG@tok@nv\endcsname{\def\PYG@tc##1{\textcolor[rgb]{0.73,0.38,0.84}{##1}}}
\expandafter\def\csname PYG@tok@s1\endcsname{\def\PYG@tc##1{\textcolor[rgb]{0.25,0.44,0.63}{##1}}}
\expandafter\def\csname PYG@tok@ch\endcsname{\let\PYG@it=\textit\def\PYG@tc##1{\textcolor[rgb]{0.25,0.50,0.56}{##1}}}
\expandafter\def\csname PYG@tok@m\endcsname{\def\PYG@tc##1{\textcolor[rgb]{0.13,0.50,0.31}{##1}}}
\expandafter\def\csname PYG@tok@gp\endcsname{\let\PYG@bf=\textbf\def\PYG@tc##1{\textcolor[rgb]{0.78,0.36,0.04}{##1}}}
\expandafter\def\csname PYG@tok@sh\endcsname{\def\PYG@tc##1{\textcolor[rgb]{0.25,0.44,0.63}{##1}}}
\expandafter\def\csname PYG@tok@ow\endcsname{\let\PYG@bf=\textbf\def\PYG@tc##1{\textcolor[rgb]{0.00,0.44,0.13}{##1}}}
\expandafter\def\csname PYG@tok@sx\endcsname{\def\PYG@tc##1{\textcolor[rgb]{0.78,0.36,0.04}{##1}}}
\expandafter\def\csname PYG@tok@bp\endcsname{\def\PYG@tc##1{\textcolor[rgb]{0.00,0.44,0.13}{##1}}}
\expandafter\def\csname PYG@tok@c1\endcsname{\let\PYG@it=\textit\def\PYG@tc##1{\textcolor[rgb]{0.25,0.50,0.56}{##1}}}
\expandafter\def\csname PYG@tok@o\endcsname{\def\PYG@tc##1{\textcolor[rgb]{0.40,0.40,0.40}{##1}}}
\expandafter\def\csname PYG@tok@kc\endcsname{\let\PYG@bf=\textbf\def\PYG@tc##1{\textcolor[rgb]{0.00,0.44,0.13}{##1}}}
\expandafter\def\csname PYG@tok@c\endcsname{\let\PYG@it=\textit\def\PYG@tc##1{\textcolor[rgb]{0.25,0.50,0.56}{##1}}}
\expandafter\def\csname PYG@tok@mf\endcsname{\def\PYG@tc##1{\textcolor[rgb]{0.13,0.50,0.31}{##1}}}
\expandafter\def\csname PYG@tok@err\endcsname{\def\PYG@bc##1{\setlength{\fboxsep}{0pt}\fcolorbox[rgb]{1.00,0.00,0.00}{1,1,1}{\strut ##1}}}
\expandafter\def\csname PYG@tok@mb\endcsname{\def\PYG@tc##1{\textcolor[rgb]{0.13,0.50,0.31}{##1}}}
\expandafter\def\csname PYG@tok@ss\endcsname{\def\PYG@tc##1{\textcolor[rgb]{0.32,0.47,0.09}{##1}}}
\expandafter\def\csname PYG@tok@sr\endcsname{\def\PYG@tc##1{\textcolor[rgb]{0.14,0.33,0.53}{##1}}}
\expandafter\def\csname PYG@tok@mo\endcsname{\def\PYG@tc##1{\textcolor[rgb]{0.13,0.50,0.31}{##1}}}
\expandafter\def\csname PYG@tok@kd\endcsname{\let\PYG@bf=\textbf\def\PYG@tc##1{\textcolor[rgb]{0.00,0.44,0.13}{##1}}}
\expandafter\def\csname PYG@tok@mi\endcsname{\def\PYG@tc##1{\textcolor[rgb]{0.13,0.50,0.31}{##1}}}
\expandafter\def\csname PYG@tok@kn\endcsname{\let\PYG@bf=\textbf\def\PYG@tc##1{\textcolor[rgb]{0.00,0.44,0.13}{##1}}}
\expandafter\def\csname PYG@tok@cpf\endcsname{\let\PYG@it=\textit\def\PYG@tc##1{\textcolor[rgb]{0.25,0.50,0.56}{##1}}}
\expandafter\def\csname PYG@tok@kr\endcsname{\let\PYG@bf=\textbf\def\PYG@tc##1{\textcolor[rgb]{0.00,0.44,0.13}{##1}}}
\expandafter\def\csname PYG@tok@s\endcsname{\def\PYG@tc##1{\textcolor[rgb]{0.25,0.44,0.63}{##1}}}
\expandafter\def\csname PYG@tok@kp\endcsname{\def\PYG@tc##1{\textcolor[rgb]{0.00,0.44,0.13}{##1}}}
\expandafter\def\csname PYG@tok@w\endcsname{\def\PYG@tc##1{\textcolor[rgb]{0.73,0.73,0.73}{##1}}}
\expandafter\def\csname PYG@tok@kt\endcsname{\def\PYG@tc##1{\textcolor[rgb]{0.56,0.13,0.00}{##1}}}
\expandafter\def\csname PYG@tok@sc\endcsname{\def\PYG@tc##1{\textcolor[rgb]{0.25,0.44,0.63}{##1}}}
\expandafter\def\csname PYG@tok@sb\endcsname{\def\PYG@tc##1{\textcolor[rgb]{0.25,0.44,0.63}{##1}}}
\expandafter\def\csname PYG@tok@k\endcsname{\let\PYG@bf=\textbf\def\PYG@tc##1{\textcolor[rgb]{0.00,0.44,0.13}{##1}}}
\expandafter\def\csname PYG@tok@se\endcsname{\let\PYG@bf=\textbf\def\PYG@tc##1{\textcolor[rgb]{0.25,0.44,0.63}{##1}}}
\expandafter\def\csname PYG@tok@sd\endcsname{\let\PYG@it=\textit\def\PYG@tc##1{\textcolor[rgb]{0.25,0.44,0.63}{##1}}}

\def\PYGZbs{\char`\\}
\def\PYGZus{\char`\_}
\def\PYGZob{\char`\{}
\def\PYGZcb{\char`\}}
\def\PYGZca{\char`\^}
\def\PYGZam{\char`\&}
\def\PYGZlt{\char`\<}
\def\PYGZgt{\char`\>}
\def\PYGZsh{\char`\#}
\def\PYGZpc{\char`\%}
\def\PYGZdl{\char`\$}
\def\PYGZhy{\char`\-}
\def\PYGZsq{\char`\'}
\def\PYGZdq{\char`\"}
\def\PYGZti{\char`\~}
% for compatibility with earlier versions
\def\PYGZat{@}
\def\PYGZlb{[}
\def\PYGZrb{]}
\makeatother

\renewcommand\PYGZsq{\textquotesingle}

\begin{document}

\maketitle
\tableofcontents
\phantomsection\label{index::doc}



\chapter{Installation}
\label{index:installation}\label{index:welcome-to-crime-modeling-and-prediction-s-documentation}

\section{Libraries}
\label{index:libraries}
NOTE: Some of this libraries may fail to install depending on the ubuntu
version you have on your machine.

\begin{Verbatim}[commandchars=\\\{\}]
\PYG{n}{sudo} \PYG{n}{apt}\PYG{o}{\PYGZhy{}}\PYG{n}{get} \PYG{n}{install} \PYG{n}{binutils} \PYG{n}{libproj}\PYG{o}{\PYGZhy{}}\PYG{n}{dev} \PYG{n}{postgresql}\PYG{o}{\PYGZhy{}}\PYG{l+m+mf}{9.4}\PYG{o}{\PYGZhy{}}\PYG{n}{postgis}\PYG{o}{\PYGZhy{}}\PYG{l+m+mf}{2.1} \PYG{n}{postgresql}\PYG{o}{\PYGZhy{}}\PYG{n}{server}\PYG{o}{\PYGZhy{}}\PYG{n}{dev}\PYG{o}{\PYGZhy{}}\PYG{l+m+mf}{9.4} \PYG{n}{postgresql}\PYG{o}{\PYGZhy{}}\PYG{n}{contrib}\PYG{o}{\PYGZhy{}}\PYG{l+m+mf}{9.4} \PYG{n}{libatlas}\PYG{o}{\PYGZhy{}}\PYG{n}{base}\PYG{o}{\PYGZhy{}}\PYG{n}{dev} \PYG{n}{libblas}\PYG{o}{\PYGZhy{}}\PYG{n}{dev}
\end{Verbatim}


\section{Clone}
\label{index:clone}
\begin{Verbatim}[commandchars=\\\{\}]
\PYG{n}{git} \PYG{n}{clone} \PYG{n}{git}\PYG{n+nd}{@github}\PYG{o}{.}\PYG{n}{com}\PYG{p}{:}\PYG{n}{jayArnel}\PYG{o}{/}\PYG{n}{crimemapping}\PYG{o}{.}\PYG{n}{git}
\PYG{n}{cd} \PYG{n}{crimemapping}
\end{Verbatim}


\section{Create Virtualenv}
\label{index:create-virtualenv}
\begin{Verbatim}[commandchars=\\\{\}]
\PYG{n}{sudo} \PYG{n}{pip} \PYG{n}{install} \PYG{n}{virtualenvwrapper}
\PYG{n}{mkvirtualenv} \PYG{o}{\PYGZlt{}}\PYG{n}{name}\PYG{o}{\PYGZgt{}}
\PYG{n}{workon} \PYG{o}{\PYGZlt{}}\PYG{n}{name}\PYG{o}{\PYGZgt{}}
\end{Verbatim}


\section{Install Package Dependencies}
\label{index:install-package-dependencies}
\begin{Verbatim}[commandchars=\\\{\}]
\PYG{n}{pip} \PYG{n}{install} \PYG{o}{\PYGZhy{}}\PYG{n}{r} \PYG{n}{requirements}\PYG{o}{.}\PYG{n}{txt}
\end{Verbatim}


\section{Create the Database}
\label{index:create-the-database}
\begin{Verbatim}[commandchars=\\\{\}]
\PYG{n}{sudo} \PYG{n}{su} \PYG{o}{\PYGZhy{}} \PYG{n}{postgres}
\PYG{n}{createdb} \PYG{o}{\PYGZlt{}}\PYG{n}{dbname}\PYG{o}{\PYGZgt{}}
\end{Verbatim}


\section{Create User for the Database}
\label{index:create-user-for-the-database}
\begin{Verbatim}[commandchars=\\\{\}]
\PYG{n}{sudo} \PYG{n}{su} \PYG{o}{\PYGZhy{}} \PYG{n}{postgres}
\PYG{n}{createuser} \PYG{o}{\PYGZhy{}}\PYG{n}{s} \PYG{o}{\PYGZhy{}}\PYG{n}{P} \PYG{o}{\PYGZlt{}}\PYG{n}{username}\PYG{o}{\PYGZgt{}}
    \PYG{o}{*}\PYG{n}{setup} \PYG{n}{password}\PYG{o}{*}
\PYG{n}{psql}
\PYG{n}{GRANT} \PYG{n}{ALL} \PYG{n}{PRIVILEGES} \PYG{n}{ON} \PYG{n}{DATABASE} \PYG{o}{\PYGZlt{}}\PYG{n}{dbname}\PYG{o}{\PYGZgt{}} \PYG{n}{TO} \PYG{o}{\PYGZlt{}}\PYG{n}{username}\PYG{o}{\PYGZgt{}}\PYG{p}{;}
\end{Verbatim}


\section{Extend Database to postgis}
\label{index:extend-database-to-postgis}
\begin{Verbatim}[commandchars=\\\{\}]
\PYG{n}{sudo} \PYG{n}{su} \PYG{o}{\PYGZhy{}} \PYG{n}{postgres}
\PYG{n}{psql} \PYG{o}{\PYGZlt{}}\PYG{n}{dbname}\PYG{o}{\PYGZgt{}}
\PYG{n}{CREATE} \PYG{n}{EXTENSION} \PYG{n}{postgis}\PYG{p}{;}
\end{Verbatim}


\section{Setup Database with Django}
\label{index:setup-database-with-django}
create \code{local\_settings.py} file in \code{crimemapping} folder

\begin{Verbatim}[commandchars=\\\{\}]
\PYG{n}{DATABASES} \PYG{o}{=} \PYG{p}{\PYGZob{}}
    \PYG{l+s+s1}{\PYGZsq{}}\PYG{l+s+s1}{default}\PYG{l+s+s1}{\PYGZsq{}}\PYG{p}{:} \PYG{p}{\PYGZob{}}
        \PYG{l+s+s1}{\PYGZsq{}}\PYG{l+s+s1}{ENGINE}\PYG{l+s+s1}{\PYGZsq{}}\PYG{p}{:} \PYG{l+s+s1}{\PYGZsq{}}\PYG{l+s+s1}{django.contrib.gis.db.backends.postgis}\PYG{l+s+s1}{\PYGZsq{}}\PYG{p}{,}
        \PYG{l+s+s1}{\PYGZsq{}}\PYG{l+s+s1}{NAME}\PYG{l+s+s1}{\PYGZsq{}}\PYG{p}{:} \PYG{l+s+s1}{\PYGZsq{}}\PYG{l+s+s1}{\PYGZlt{}dbname\PYGZgt{}}\PYG{l+s+s1}{\PYGZsq{}}\PYG{p}{,}
        \PYG{l+s+s1}{\PYGZsq{}}\PYG{l+s+s1}{USER}\PYG{l+s+s1}{\PYGZsq{}}\PYG{p}{:} \PYG{l+s+s1}{\PYGZsq{}}\PYG{l+s+s1}{\PYGZlt{}username\PYGZgt{}}\PYG{l+s+s1}{\PYGZsq{}}\PYG{p}{,}
        \PYG{l+s+s1}{\PYGZsq{}}\PYG{l+s+s1}{PASSWORD}\PYG{l+s+s1}{\PYGZsq{}}\PYG{p}{:} \PYG{l+s+s1}{\PYGZsq{}}\PYG{l+s+s1}{\PYGZlt{}password\PYGZgt{}}\PYG{l+s+s1}{\PYGZsq{}}\PYG{p}{,}
        \PYG{l+s+s1}{\PYGZsq{}}\PYG{l+s+s1}{HOST}\PYG{l+s+s1}{\PYGZsq{}}\PYG{p}{:} \PYG{l+s+s1}{\PYGZsq{}}\PYG{l+s+s1}{localhost}\PYG{l+s+s1}{\PYGZsq{}}\PYG{p}{,}
        \PYG{l+s+s1}{\PYGZsq{}}\PYG{l+s+s1}{PORT}\PYG{l+s+s1}{\PYGZsq{}}\PYG{p}{:} \PYG{l+s+s1}{\PYGZsq{}}\PYG{l+s+s1}{\PYGZsq{}}\PYG{p}{,}
    \PYG{p}{\PYGZcb{}}
\PYG{p}{\PYGZcb{}}
\end{Verbatim}


\section{Run Migrations}
\label{index:run-migrations}
make sure your virtualenv \code{\textless{}env\textgreater{}} is activated by calling
\code{workon \textless{}env\textgreater{}}

\begin{Verbatim}[commandchars=\\\{\}]
\PYG{n}{python} \PYG{n}{manage}\PYG{o}{.}\PYG{n}{py} \PYG{n}{migrate}
\end{Verbatim}


\section{Collect static files}
\label{index:collect-static-files}
\begin{Verbatim}[commandchars=\\\{\}]
\PYG{n}{python} \PYG{n}{manage}\PYG{o}{.}\PYG{n}{py} \PYG{n}{collectstatic} \PYG{o}{\PYGZhy{}}\PYG{n}{l} \PYG{o}{\PYGZhy{}}\PYG{o}{\PYGZhy{}}\PYG{n}{no}\PYG{o}{\PYGZhy{}}\PYG{n+nb}{input}
\end{Verbatim}


\section{Setup Nginx}
\label{index:setup-nginx}
setup a local server in your machine via Nginx


\section{Install nginx}
\label{index:install-nginx}
\begin{Verbatim}[commandchars=\\\{\}]
\PYG{n}{sudo} \PYG{n}{apt}\PYG{o}{\PYGZhy{}}\PYG{n}{get} \PYG{n}{install} \PYG{n}{nginx}
\end{Verbatim}


\section{Setup Nginx configuration}
\label{index:setup-nginx-configuration}
create configuration file

\begin{Verbatim}[commandchars=\\\{\}]
\PYG{n}{sudo} \PYG{n}{nano} \PYG{o}{/}\PYG{n}{etc}\PYG{o}{/}\PYG{n}{nginx}\PYG{o}{/}\PYG{n}{sites}\PYG{o}{\PYGZhy{}}\PYG{n}{available}\PYG{o}{/}\PYG{n}{crimemapping}
\end{Verbatim}

copy-paste configuration

\begin{Verbatim}[commandchars=\\\{\}]
upstream localhost \PYGZob{}
    server localhost:8000;
\PYGZcb{}
server \PYGZob{}
    listen 0.0.0.0:80;
    server\PYGZus{}name localhost;

    keepalive\PYGZus{}timeout 5;
    \PYGZsh{} path for static files
    root /path/to/project/storage; \PYGZsh{}! important! update this path
    location / \PYGZob{}
        proxy\PYGZus{}set\PYGZus{}header X\PYGZhy{}Forwarded\PYGZhy{}For \PYGZdl{}proxy\PYGZus{}add\PYGZus{}x\PYGZus{}forwarded\PYGZus{}for;
        proxy\PYGZus{}set\PYGZus{}header Host \PYGZdl{}http\PYGZus{}host;
        proxy\PYGZus{}redirect off;
        if (!\PYGZhy{}f \PYGZdl{}request\PYGZus{}filename) \PYGZob{}
            proxy\PYGZus{}pass http://localhost;
            break;
        \PYGZcb{}
    \PYGZcb{}
\PYGZcb{}
\end{Verbatim}

enable configuration
\code{shell sudo ln -s /etc/nginx/sites-available/crimemapping /etc/nginx/sites-enabled/crimemapping {}`{}` check for errors, then restart nginx}
nginx sudo nginx -t sudo service nginx restart {}`{}`{}`


\section{Starting the Web App}
\label{index:starting-the-web-app}
\begin{Verbatim}[commandchars=\\\{\}]
\PYG{n}{python} \PYG{n}{manage}\PYG{o}{.}\PYG{n}{py} \PYG{n}{runserver}
\end{Verbatim}

Open ‘localhost’ on your web browser


\section{Running the Model}
\label{index:running-the-model}
\begin{Verbatim}[commandchars=\\\{\}]
\PYG{n}{python} \PYG{n}{manage}\PYG{o}{.}\PYG{n}{py} \PYG{n}{shell}
\end{Verbatim}

\begin{Verbatim}[commandchars=\\\{\}]
\PYG{k+kn}{from} \PYG{n+nn}{crimeprediction} \PYG{k}{import} \PYG{n}{network}

\PYG{n}{network}\PYG{o}{.}\PYG{n}{run\PYGZus{}network}\PYG{p}{(}\PYG{l+m+mi}{1000}\PYG{p}{,} \PYG{l+s+s1}{\PYGZsq{}}\PYG{l+s+s1}{monthly}\PYG{l+s+s1}{\PYGZsq{}}\PYG{p}{,} \PYG{n}{crime\PYGZus{}type}\PYG{o}{=}\PYG{k+kc}{None}\PYG{p}{,} \PYG{n}{seasonal}\PYG{o}{=}\PYG{k+kc}{True}\PYG{p}{)}
\end{Verbatim}


\chapter{API}
\label{index:api}

\section{core package}
\label{api/core::doc}\label{api/core:core-package}

\subsection{Subpackages}
\label{api/core:subpackages}

\subsubsection{core.finders package}
\label{api/core.finders:core-finders-package}\label{api/core.finders::doc}

\paragraph{Submodules}
\label{api/core.finders:submodules}

\paragraph{core.finders.mustache\_template module}
\label{api/core.finders:module-core.finders.mustache_template}\label{api/core.finders:core-finders-mustache-template-module}\index{core.finders.mustache\_template (module)}\index{MustacheTemplateFinder (class in core.finders.mustache\_template)}

\begin{fulllineitems}
\phantomsection\label{api/core.finders:core.finders.mustache_template.MustacheTemplateFinder}\pysiglinewithargsret{\strong{class }\code{core.finders.mustache\_template.}\bfcode{MustacheTemplateFinder}}{\emph{app\_names=None}, \emph{*args}, \emph{**kwargs}}{}
Bases: \code{django.contrib.staticfiles.finders.AppDirectoriesFinder}

Static files finder to locate mustache template files.
\index{source\_dir (core.finders.mustache\_template.MustacheTemplateFinder attribute)}

\begin{fulllineitems}
\phantomsection\label{api/core.finders:core.finders.mustache_template.MustacheTemplateFinder.source_dir}\pysigline{\bfcode{source\_dir}\strong{ = `mustachetemplates'}}
\end{fulllineitems}


\end{fulllineitems}



\section{crime package}
\label{api/crime:crime-package}\label{api/crime::doc}

\subsection{Submodules}
\label{api/crime:submodules}

\subsection{crime.admin module}
\label{api/crime:module-crime.admin}\label{api/crime:crime-admin-module}\index{crime.admin (module)}
Admin configuration for the Crime module
\index{CriminalRecordAdmin (class in crime.admin)}

\begin{fulllineitems}
\phantomsection\label{api/crime:crime.admin.CriminalRecordAdmin}\pysiglinewithargsret{\strong{class }\code{crime.admin.}\bfcode{CriminalRecordAdmin}}{\emph{model}, \emph{admin\_site}}{}
Bases: \code{django.contrib.admin.options.ModelAdmin}

display CriminalRecord Model in admin page
\index{date\_hierarchy (crime.admin.CriminalRecordAdmin attribute)}

\begin{fulllineitems}
\phantomsection\label{api/crime:crime.admin.CriminalRecordAdmin.date_hierarchy}\pysigline{\bfcode{date\_hierarchy}\strong{ = `date'}}
\end{fulllineitems}

\index{list\_display (crime.admin.CriminalRecordAdmin attribute)}

\begin{fulllineitems}
\phantomsection\label{api/crime:crime.admin.CriminalRecordAdmin.list_display}\pysigline{\bfcode{list\_display}\strong{ = (`primary\_type', `crime\_description', `location\_description', `date', `latitude', `longitude')}}
\end{fulllineitems}

\index{list\_filter (crime.admin.CriminalRecordAdmin attribute)}

\begin{fulllineitems}
\phantomsection\label{api/crime:crime.admin.CriminalRecordAdmin.list_filter}\pysigline{\bfcode{list\_filter}\strong{ = (`date',)}}
\end{fulllineitems}

\index{media (crime.admin.CriminalRecordAdmin attribute)}

\begin{fulllineitems}
\phantomsection\label{api/crime:crime.admin.CriminalRecordAdmin.media}\pysigline{\bfcode{media}}
\end{fulllineitems}

\index{search\_fields (crime.admin.CriminalRecordAdmin attribute)}

\begin{fulllineitems}
\phantomsection\label{api/crime:crime.admin.CriminalRecordAdmin.search_fields}\pysigline{\bfcode{search\_fields}\strong{ = (`primary\_type', `crime\_description', `location\_description', `date', `latitude', `longitude')}}
\end{fulllineitems}


\end{fulllineitems}



\subsection{crime.api module}
\label{api/crime:crime-api-module}\label{api/crime:module-crime.api}\index{crime.api (module)}
API resources for the Crime module
\index{CriminalRecordResource (class in crime.api)}

\begin{fulllineitems}
\phantomsection\label{api/crime:crime.api.CriminalRecordResource}\pysiglinewithargsret{\strong{class }\code{crime.api.}\bfcode{CriminalRecordResource}}{\emph{api\_name=None}}{}
Bases: \code{tastypie.resources.ModelResource}

Api resource for CriminalRecord Model
\index{CriminalRecordResource.Meta (class in crime.api)}

\begin{fulllineitems}
\phantomsection\label{api/crime:crime.api.CriminalRecordResource.Meta}\pysigline{\strong{class }\bfcode{Meta}}~\index{filtering (crime.api.CriminalRecordResource.Meta attribute)}

\begin{fulllineitems}
\phantomsection\label{api/crime:crime.api.CriminalRecordResource.Meta.filtering}\pysigline{\bfcode{filtering}\strong{ = \{`date': {[}'range', `exact', `gt', `gte', `lt', `lte'{]}, `primary\_type': {[}'exact', `in'{]}\}}}
\end{fulllineitems}

\index{object\_class (crime.api.CriminalRecordResource.Meta attribute)}

\begin{fulllineitems}
\phantomsection\label{api/crime:crime.api.CriminalRecordResource.Meta.object_class}\pysigline{\bfcode{object\_class}}
alias of \code{CriminalRecord}

\end{fulllineitems}

\index{queryset (crime.api.CriminalRecordResource.Meta attribute)}

\begin{fulllineitems}
\phantomsection\label{api/crime:crime.api.CriminalRecordResource.Meta.queryset}\pysigline{\bfcode{queryset}\strong{ = {[}\textless{}CriminalRecord: CriminalRecord object\textgreater{}, \textless{}CriminalRecord: CriminalRecord object\textgreater{}, \textless{}CriminalRecord: CriminalRecord object\textgreater{}, \textless{}CriminalRecord: CriminalRecord object\textgreater{}, \textless{}CriminalRecord: CriminalRecord object\textgreater{}, \textless{}CriminalRecord: CriminalRecord object\textgreater{}, \textless{}CriminalRecord: CriminalRecord object\textgreater{}, \textless{}CriminalRecord: CriminalRecord object\textgreater{}, \textless{}CriminalRecord: CriminalRecord object\textgreater{}, \textless{}CriminalRecord: CriminalRecord object\textgreater{}, \textless{}CriminalRecord: CriminalRecord object\textgreater{}, \textless{}CriminalRecord: CriminalRecord object\textgreater{}, \textless{}CriminalRecord: CriminalRecord object\textgreater{}, \textless{}CriminalRecord: CriminalRecord object\textgreater{}, \textless{}CriminalRecord: CriminalRecord object\textgreater{}, \textless{}CriminalRecord: CriminalRecord object\textgreater{}, \textless{}CriminalRecord: CriminalRecord object\textgreater{}, \textless{}CriminalRecord: CriminalRecord object\textgreater{}, \textless{}CriminalRecord: CriminalRecord object\textgreater{}, \textless{}CriminalRecord: CriminalRecord object\textgreater{}, `...(remaining elements truncated)...'{]}}}
\end{fulllineitems}

\index{resource\_name (crime.api.CriminalRecordResource.Meta attribute)}

\begin{fulllineitems}
\phantomsection\label{api/crime:crime.api.CriminalRecordResource.Meta.resource_name}\pysigline{\bfcode{resource\_name}\strong{ = `criminalrecord'}}
\end{fulllineitems}


\end{fulllineitems}

\index{base\_fields (crime.api.CriminalRecordResource attribute)}

\begin{fulllineitems}
\phantomsection\label{api/crime:crime.api.CriminalRecordResource.base_fields}\pysigline{\code{CriminalRecordResource.}\bfcode{base\_fields}\strong{ = \{`primary\_type': \textless{}tastypie.fields.CharField object at 0xaa9a3fcc\textgreater{}, `beat': \textless{}tastypie.fields.CharField object at 0xaa9ab12c\textgreater{}, `y\_coordinate': \textless{}tastypie.fields.IntegerField object at 0xaa9ab2ac\textgreater{}, `year': \textless{}tastypie.fields.IntegerField object at 0xaa9ab2ec\textgreater{}, `has\_arrested': \textless{}tastypie.fields.BooleanField object at 0xaa9ab0ac\textgreater{}, `case\_number': \textless{}tastypie.fields.CharField object at 0xaa9a3eec\textgreater{}, `date': \textless{}tastypie.fields.DateTimeField object at 0xaa9a3f0c\textgreater{}, `ward': \textless{}tastypie.fields.IntegerField object at 0xaa9ab1ac\textgreater{}, `x\_coordinate': \textless{}tastypie.fields.IntegerField object at 0xaa9ab26c\textgreater{}, `crime\_description': \textless{}tastypie.fields.CharField object at 0xaa9ab02c\textgreater{}, u'id': \textless{}tastypie.fields.IntegerField object at 0xaa9a3eac\textgreater{}, `district': \textless{}tastypie.fields.CharField object at 0xaa9ab16c\textgreater{}, `iucr': \textless{}tastypie.fields.CharField object at 0xaa9a3f8c\textgreater{}, `longitude': \textless{}tastypie.fields.FloatField object at 0xaa9ab3ac\textgreater{}, `location\_description': \textless{}tastypie.fields.CharField object at 0xaa9ab06c\textgreater{}, `case\_id': \textless{}tastypie.fields.IntegerField object at 0xaa99bbac\textgreater{}, `is\_domestic': \textless{}tastypie.fields.BooleanField object at 0xaa9ab0ec\textgreater{}, `location': \textless{}tastypie.fields.CharField object at 0xaa9ab3ec\textgreater{}, `community\_area': \textless{}tastypie.fields.CharField object at 0xaa9ab1ec\textgreater{}, `latitude': \textless{}tastypie.fields.FloatField object at 0xaa9ab36c\textgreater{}, `updated\_on': \textless{}tastypie.fields.DateTimeField object at 0xaa9ab32c\textgreater{}, `fbi\_code': \textless{}tastypie.fields.CharField object at 0xaa9ab22c\textgreater{}, `block': \textless{}tastypie.fields.CharField object at 0xaa9a3f4c\textgreater{}, u'resource\_uri': \textless{}tastypie.fields.CharField object at 0xaa9a3e2c\textgreater{}\}}}
\end{fulllineitems}

\index{declared\_fields (crime.api.CriminalRecordResource attribute)}

\begin{fulllineitems}
\phantomsection\label{api/crime:crime.api.CriminalRecordResource.declared_fields}\pysigline{\code{CriminalRecordResource.}\bfcode{declared\_fields}\strong{ = \{\}}}
\end{fulllineitems}


\end{fulllineitems}



\subsection{crime.models module}
\label{api/crime:module-crime.models}\label{api/crime:crime-models-module}\index{crime.models (module)}
Models for the Crime module
\index{CriminalRecord (class in crime.models)}

\begin{fulllineitems}
\phantomsection\label{api/crime:crime.models.CriminalRecord}\pysiglinewithargsret{\strong{class }\code{crime.models.}\bfcode{CriminalRecord}}{\emph{*args}, \emph{**kwargs}}{}
Bases: \code{django.db.models.base.Model}

a model for a single criminal record
\index{CriminalRecord.DoesNotExist}

\begin{fulllineitems}
\phantomsection\label{api/crime:crime.models.CriminalRecord.DoesNotExist}\pysigline{\strong{exception }\bfcode{DoesNotExist}}
Bases: \code{django.core.exceptions.ObjectDoesNotExist}

\end{fulllineitems}

\index{CriminalRecord.MultipleObjectsReturned}

\begin{fulllineitems}
\phantomsection\label{api/crime:crime.models.CriminalRecord.MultipleObjectsReturned}\pysigline{\strong{exception }\code{CriminalRecord.}\bfcode{MultipleObjectsReturned}}
Bases: \code{django.core.exceptions.MultipleObjectsReturned}

\end{fulllineitems}

\index{get\_next\_by\_date() (crime.models.CriminalRecord method)}

\begin{fulllineitems}
\phantomsection\label{api/crime:crime.models.CriminalRecord.get_next_by_date}\pysiglinewithargsret{\code{CriminalRecord.}\bfcode{get\_next\_by\_date}}{\emph{*moreargs}, \emph{**morekwargs}}{}
\end{fulllineitems}

\index{get\_next\_by\_updated\_on() (crime.models.CriminalRecord method)}

\begin{fulllineitems}
\phantomsection\label{api/crime:crime.models.CriminalRecord.get_next_by_updated_on}\pysiglinewithargsret{\code{CriminalRecord.}\bfcode{get\_next\_by\_updated\_on}}{\emph{*moreargs}, \emph{**morekwargs}}{}
\end{fulllineitems}

\index{get\_previous\_by\_date() (crime.models.CriminalRecord method)}

\begin{fulllineitems}
\phantomsection\label{api/crime:crime.models.CriminalRecord.get_previous_by_date}\pysiglinewithargsret{\code{CriminalRecord.}\bfcode{get\_previous\_by\_date}}{\emph{*moreargs}, \emph{**morekwargs}}{}
\end{fulllineitems}

\index{get\_previous\_by\_updated\_on() (crime.models.CriminalRecord method)}

\begin{fulllineitems}
\phantomsection\label{api/crime:crime.models.CriminalRecord.get_previous_by_updated_on}\pysiglinewithargsret{\code{CriminalRecord.}\bfcode{get\_previous\_by\_updated\_on}}{\emph{*moreargs}, \emph{**morekwargs}}{}
\end{fulllineitems}

\index{location (crime.models.CriminalRecord attribute)}

\begin{fulllineitems}
\phantomsection\label{api/crime:crime.models.CriminalRecord.location}\pysigline{\code{CriminalRecord.}\bfcode{location}}
\end{fulllineitems}

\index{objects (crime.models.CriminalRecord attribute)}

\begin{fulllineitems}
\phantomsection\label{api/crime:crime.models.CriminalRecord.objects}\pysigline{\code{CriminalRecord.}\bfcode{objects}\strong{ = \textless{}django.db.models.manager.Manager object\textgreater{}}}
\end{fulllineitems}


\end{fulllineitems}



\subsection{crime.views module}
\label{api/crime:crime-views-module}\label{api/crime:module-crime.views}\index{crime.views (module)}
Views for the Crime module
\index{FetchCrimesView (class in crime.views)}

\begin{fulllineitems}
\phantomsection\label{api/crime:crime.views.FetchCrimesView}\pysiglinewithargsret{\strong{class }\code{crime.views.}\bfcode{FetchCrimesView}}{\emph{**kwargs}}{}
Bases: \code{django.views.generic.base.View}

Fetch criminal records from online data portal
\index{get() (crime.views.FetchCrimesView method)}

\begin{fulllineitems}
\phantomsection\label{api/crime:crime.views.FetchCrimesView.get}\pysiglinewithargsret{\bfcode{get}}{\emph{request}, \emph{*args}, \emph{**kwargs}}{}
Get criminal records from the dataset and save them the database as CriminalRecord objects

\end{fulllineitems}

\index{get\_from\_dict() (crime.views.FetchCrimesView method)}

\begin{fulllineitems}
\phantomsection\label{api/crime:crime.views.FetchCrimesView.get_from_dict}\pysiglinewithargsret{\bfcode{get\_from\_dict}}{\emph{dic}, \emph{key}}{}
get an value from the dictionary using a key but instead of None
return an empty string if it does not exist
\begin{quote}\begin{description}
\item[{Parameters}] \leavevmode\begin{itemize}
\item {} 
\textbf{\texttt{dic}} -- the dictionary to be search

\item {} 
\textbf{\texttt{key}} -- key of the value to be obtained

\end{itemize}

\item[{Return type}] \leavevmode
value of the key in the dictionary or empty string

\end{description}\end{quote}

\end{fulllineitems}

\index{to\_int() (crime.views.FetchCrimesView method)}

\begin{fulllineitems}
\phantomsection\label{api/crime:crime.views.FetchCrimesView.to_int}\pysiglinewithargsret{\bfcode{to\_int}}{\emph{data}}{}
converts data to interger, return None instead of number
\begin{quote}\begin{description}
\item[{Parameters}] \leavevmode
\textbf{\texttt{data}} -- data to be converted

\item[{Return type}] \leavevmode
interger converstion of the data or None

\end{description}\end{quote}

\end{fulllineitems}


\end{fulllineitems}



\section{crimeprediction package}
\label{api/crimeprediction:crimeprediction-package}\label{api/crimeprediction::doc}

\subsection{Submodules}
\label{api/crimeprediction:submodules}

\subsection{crimeprediction.error\_analysis module}
\label{api/crimeprediction:crimeprediction-error-analysis-module}\label{api/crimeprediction:module-crimeprediction.error_analysis}\index{crimeprediction.error\_analysis (module)}\index{build\_model() (in module crimeprediction.error\_analysis)}

\begin{fulllineitems}
\phantomsection\label{api/crimeprediction:crimeprediction.error_analysis.build_model}\pysiglinewithargsret{\code{crimeprediction.error\_analysis.}\bfcode{build\_model}}{\emph{dim}, \emph{X}}{}
build function required by Scikit-Learn's learning curve function
\begin{quote}\begin{description}
\item[{Parameters}] \leavevmode\begin{itemize}
\item {} 
\textbf{\texttt{dim}} -- dimension of the inputs

\item {} 
\textbf{\texttt{X}} -- data vector

\end{itemize}

\item[{Return type}] \leavevmode
contructed model

\end{description}\end{quote}

\end{fulllineitems}

\index{plot\_learning\_curve() (in module crimeprediction.error\_analysis)}

\begin{fulllineitems}
\phantomsection\label{api/crimeprediction:crimeprediction.error_analysis.plot_learning_curve}\pysiglinewithargsret{\code{crimeprediction.error\_analysis.}\bfcode{plot\_learning\_curve}}{\emph{grid\_size}, \emph{period}, \emph{crime\_type=None}, \emph{seasonal=False}}{}~\begin{description}
\item[{Plots the learning curve using Scikit-Learn's learning\_curve function,}] \leavevmode
plots the graph using matplotlib and save sit
\begin{quote}\begin{description}
\item[{param grid\_size}] \leavevmode
size of the cell dimension for the grid

\item[{param period}] \leavevmode
timestep of crime data

\item[{param crime\_type}] \leavevmode
type of crime to be trained, None value will
train all

\item[{param seasonal}] \leavevmode
implement seasonality or not

\end{description}\end{quote}

\end{description}

\end{fulllineitems}



\subsection{crimeprediction.experiments module}
\label{api/crimeprediction:crimeprediction-experiments-module}\label{api/crimeprediction:module-crimeprediction.experiments}\index{crimeprediction.experiments (module)}\index{run\_experiments() (in module crimeprediction.experiments)}

\begin{fulllineitems}
\phantomsection\label{api/crimeprediction:crimeprediction.experiments.run_experiments}\pysiglinewithargsret{\code{crimeprediction.experiments.}\bfcode{run\_experiments}}{}{}
Perfrom all experiments by running network under all possible conditions

\end{fulllineitems}



\subsection{crimeprediction.network module}
\label{api/crimeprediction:module-crimeprediction.network}\label{api/crimeprediction:crimeprediction-network-module}\index{crimeprediction.network (module)}\index{get\_trained\_model() (in module crimeprediction.network)}

\begin{fulllineitems}
\phantomsection\label{api/crimeprediction:crimeprediction.network.get_trained_model}\pysiglinewithargsret{\code{crimeprediction.network.}\bfcode{get\_trained\_model}}{}{}
reconstruct trained model from saved files
\begin{quote}\begin{description}
\item[{Return type}] \leavevmode
a tuple of the model constructed and a yaml string of parameters

\end{description}\end{quote}

used

\end{fulllineitems}

\index{predict\_next() (in module crimeprediction.network)}

\begin{fulllineitems}
\phantomsection\label{api/crimeprediction:crimeprediction.network.predict_next}\pysiglinewithargsret{\code{crimeprediction.network.}\bfcode{predict\_next}}{\emph{model}, \emph{**params}}{}
predicts next crime hotspots
\begin{quote}\begin{description}
\item[{Parameters}] \leavevmode\begin{itemize}
\item {} 
\textbf{\texttt{model}} -- the model to be used for prediction

\item {} 
\textbf{\texttt{**params}} -- 
a yaml string of the parameters used by the model


\end{itemize}

\end{description}\end{quote}

\end{fulllineitems}

\index{run\_network() (in module crimeprediction.network)}

\begin{fulllineitems}
\phantomsection\label{api/crimeprediction:crimeprediction.network.run_network}\pysiglinewithargsret{\code{crimeprediction.network.}\bfcode{run\_network}}{\emph{grid\_size}, \emph{period}, \emph{crime\_type=None}, \emph{seasonal=False}}{}
Build, train and run LSTM network
\begin{quote}\begin{description}
\item[{Parameters}] \leavevmode\begin{itemize}
\item {} 
\textbf{\texttt{grid\_size}} -- size of the cell dimension for the grid

\item {} 
\textbf{\texttt{period}} -- timestep of crime data

\item {} 
\textbf{\texttt{crime\_type}} -- type of crime to be trained, None value will
train all

\item {} 
\textbf{\texttt{seasonal}} -- implement seasonality or not

\end{itemize}

\end{description}\end{quote}

\end{fulllineitems}

\index{save\_trained\_model() (in module crimeprediction.network)}

\begin{fulllineitems}
\phantomsection\label{api/crimeprediction:crimeprediction.network.save_trained_model}\pysiglinewithargsret{\code{crimeprediction.network.}\bfcode{save\_trained\_model}}{\emph{model}, \emph{params\_string}}{}
saves trained model to directory and files depending on settings variables
\begin{quote}\begin{description}
\item[{Parameters}] \leavevmode\begin{itemize}
\item {} 
\textbf{\texttt{model}} -- model to be saved

\item {} 
\textbf{\texttt{params\_string}} -- a yaml string of parameters used for the model: crime\_type, period, grid\_size and seasonality

\end{itemize}

\end{description}\end{quote}

\end{fulllineitems}



\subsection{crimeprediction.vectorize module}
\label{api/crimeprediction:module-crimeprediction.vectorize}\label{api/crimeprediction:crimeprediction-vectorize-module}\index{crimeprediction.vectorize (module)}\index{vectorize() (in module crimeprediction.vectorize)}

\begin{fulllineitems}
\phantomsection\label{api/crimeprediction:crimeprediction.vectorize.vectorize}\pysiglinewithargsret{\code{crimeprediction.vectorize.}\bfcode{vectorize}}{\emph{grid\_size}, \emph{period}, \emph{crime\_type=None}, \emph{seasonal=False}, \emph{new=False}}{}~\begin{description}
\item[{Vectorize crime data fetched from the database. The vector is saved to a}] \leavevmode
pickle file and then retrieved later

\end{description}
\begin{quote}\begin{description}
\item[{Parameters}] \leavevmode\begin{itemize}
\item {} 
\textbf{\texttt{grid\_size}} -- size of the cell dimension for the grid

\item {} 
\textbf{\texttt{period}} -- timestep of crime data

\item {} 
\textbf{\texttt{crime\_type}} -- type of crime to be trained, None value will
train all

\item {} 
\textbf{\texttt{seasonal}} -- implement seasonality or not

\item {} 
\textbf{\texttt{new}} -- will override the old one pickle file if and if True

\end{itemize}

\item[{Raises}] \leavevmode
EnvironmentError, NotImplementedError

\item[{Return type}] \leavevmode
returns the resulting vector

\end{description}\end{quote}

\end{fulllineitems}

\index{vectorize\_weekly() (in module crimeprediction.vectorize)}

\begin{fulllineitems}
\phantomsection\label{api/crimeprediction:crimeprediction.vectorize.vectorize_weekly}\pysiglinewithargsret{\code{crimeprediction.vectorize.}\bfcode{vectorize\_weekly}}{\emph{grid}, \emph{crime\_type=None}, \emph{seasonal=False}}{}
Special vectorization for weekly data
\begin{quote}\begin{description}
\item[{Parameters}] \leavevmode\begin{itemize}
\item {} 
\textbf{\texttt{grid\_size}} -- size of the cell dimension for the grid

\item {} 
\textbf{\texttt{crime\_type}} -- type of crime to be trained, None value will
train all

\item {} 
\textbf{\texttt{seasonal}} -- implement seasonality or not

\end{itemize}

\item[{Return type}] \leavevmode
returns the resulting vector

\end{description}\end{quote}

\end{fulllineitems}



\subsection{crimeprediction.views module}
\label{api/crimeprediction:module-crimeprediction.views}\label{api/crimeprediction:crimeprediction-views-module}\index{crimeprediction.views (module)}\index{DashboardView (class in crimeprediction.views)}

\begin{fulllineitems}
\phantomsection\label{api/crimeprediction:crimeprediction.views.DashboardView}\pysiglinewithargsret{\strong{class }\code{crimeprediction.views.}\bfcode{DashboardView}}{\emph{**kwargs}}{}
Bases: \code{django.views.generic.base.TemplateView}

View for viewing dashboard or interface for the model
\index{get\_context\_data() (crimeprediction.views.DashboardView method)}

\begin{fulllineitems}
\phantomsection\label{api/crimeprediction:crimeprediction.views.DashboardView.get_context_data}\pysiglinewithargsret{\bfcode{get\_context\_data}}{\emph{**kwargs}}{}
\end{fulllineitems}

\index{template\_name (crimeprediction.views.DashboardView attribute)}

\begin{fulllineitems}
\phantomsection\label{api/crimeprediction:crimeprediction.views.DashboardView.template_name}\pysigline{\bfcode{template\_name}\strong{ = `map/dashboard.html'}}
\end{fulllineitems}


\end{fulllineitems}

\index{HomeView (class in crimeprediction.views)}

\begin{fulllineitems}
\phantomsection\label{api/crimeprediction:crimeprediction.views.HomeView}\pysiglinewithargsret{\strong{class }\code{crimeprediction.views.}\bfcode{HomeView}}{\emph{**kwargs}}{}
Bases: \code{django.views.generic.base.TemplateView}

View for the home page
\index{template\_name (crimeprediction.views.HomeView attribute)}

\begin{fulllineitems}
\phantomsection\label{api/crimeprediction:crimeprediction.views.HomeView.template_name}\pysigline{\bfcode{template\_name}\strong{ = `home/home.html'}}
\end{fulllineitems}


\end{fulllineitems}

\index{MapView (class in crimeprediction.views)}

\begin{fulllineitems}
\phantomsection\label{api/crimeprediction:crimeprediction.views.MapView}\pysiglinewithargsret{\strong{class }\code{crimeprediction.views.}\bfcode{MapView}}{\emph{**kwargs}}{}
Bases: \code{django.views.generic.base.TemplateView}

View for showing the map
\index{get\_context\_data() (crimeprediction.views.MapView method)}

\begin{fulllineitems}
\phantomsection\label{api/crimeprediction:crimeprediction.views.MapView.get_context_data}\pysiglinewithargsret{\bfcode{get\_context\_data}}{\emph{**kwargs}}{}
\end{fulllineitems}

\index{template\_name (crimeprediction.views.MapView attribute)}

\begin{fulllineitems}
\phantomsection\label{api/crimeprediction:crimeprediction.views.MapView.template_name}\pysigline{\bfcode{template\_name}\strong{ = `map/map.html'}}
\end{fulllineitems}


\end{fulllineitems}

\index{PredictView (class in crimeprediction.views)}

\begin{fulllineitems}
\phantomsection\label{api/crimeprediction:crimeprediction.views.PredictView}\pysiglinewithargsret{\strong{class }\code{crimeprediction.views.}\bfcode{PredictView}}{\emph{**kwargs}}{}
Bases: \code{django.views.generic.base.View}

View for generating prediction of next hotspot
\index{get() (crimeprediction.views.PredictView method)}

\begin{fulllineitems}
\phantomsection\label{api/crimeprediction:crimeprediction.views.PredictView.get}\pysiglinewithargsret{\bfcode{get}}{\emph{request}, \emph{*args}, \emph{**kwargs}}{}
\end{fulllineitems}


\end{fulllineitems}

\index{TrainView (class in crimeprediction.views)}

\begin{fulllineitems}
\phantomsection\label{api/crimeprediction:crimeprediction.views.TrainView}\pysiglinewithargsret{\strong{class }\code{crimeprediction.views.}\bfcode{TrainView}}{\emph{**kwargs}}{}
Bases: \code{django.views.generic.base.View}

View for training a model whose parameters are chosen via web interface
\index{get() (crimeprediction.views.TrainView method)}

\begin{fulllineitems}
\phantomsection\label{api/crimeprediction:crimeprediction.views.TrainView.get}\pysiglinewithargsret{\bfcode{get}}{\emph{request}, \emph{*args}, \emph{**kwargs}}{}
\end{fulllineitems}


\end{fulllineitems}



\section{map package}
\label{api/map::doc}\label{api/map:map-package}

\subsection{Submodules}
\label{api/map:submodules}

\subsection{map.api module}
\label{api/map:map-api-module}\label{api/map:module-map.api}\index{map.api (module)}\index{CityBorderDetailResource (class in map.api)}

\begin{fulllineitems}
\phantomsection\label{api/map:map.api.CityBorderDetailResource}\pysiglinewithargsret{\strong{class }\code{map.api.}\bfcode{CityBorderDetailResource}}{\emph{api\_name=None}}{}
Bases: \code{tastypie.resources.ModelResource}

API resource for the details of the city border
\index{CityBorderDetailResource.Meta (class in map.api)}

\begin{fulllineitems}
\phantomsection\label{api/map:map.api.CityBorderDetailResource.Meta}\pysigline{\strong{class }\bfcode{Meta}}~\index{excludes (map.api.CityBorderDetailResource.Meta attribute)}

\begin{fulllineitems}
\phantomsection\label{api/map:map.api.CityBorderDetailResource.Meta.excludes}\pysigline{\bfcode{excludes}\strong{ = {[}'geom'{]}}}
\end{fulllineitems}

\index{filtering (map.api.CityBorderDetailResource.Meta attribute)}

\begin{fulllineitems}
\phantomsection\label{api/map:map.api.CityBorderDetailResource.Meta.filtering}\pysigline{\bfcode{filtering}\strong{ = \{`name': `exact'\}}}
\end{fulllineitems}

\index{object\_class (map.api.CityBorderDetailResource.Meta attribute)}

\begin{fulllineitems}
\phantomsection\label{api/map:map.api.CityBorderDetailResource.Meta.object_class}\pysigline{\bfcode{object\_class}}
alias of \code{CityBorder}

\end{fulllineitems}

\index{queryset (map.api.CityBorderDetailResource.Meta attribute)}

\begin{fulllineitems}
\phantomsection\label{api/map:map.api.CityBorderDetailResource.Meta.queryset}\pysigline{\bfcode{queryset}\strong{ = {[}\textless{}CityBorder: Chicago\textgreater{}{]}}}
\end{fulllineitems}

\index{resource\_name (map.api.CityBorderDetailResource.Meta attribute)}

\begin{fulllineitems}
\phantomsection\label{api/map:map.api.CityBorderDetailResource.Meta.resource_name}\pysigline{\bfcode{resource\_name}\strong{ = `cityborder-detail'}}
\end{fulllineitems}


\end{fulllineitems}

\index{base\_fields (map.api.CityBorderDetailResource attribute)}

\begin{fulllineitems}
\phantomsection\label{api/map:map.api.CityBorderDetailResource.base_fields}\pysigline{\code{CityBorderDetailResource.}\bfcode{base\_fields}\strong{ = \{`shape\_area': \textless{}tastypie.fields.FloatField object at 0xaa96b72c\textgreater{}, `name': \textless{}tastypie.fields.CharField object at 0xaa96b70c\textgreater{}, `objectid': \textless{}tastypie.fields.IntegerField object at 0xaa96736c\textgreater{}, u'id': \textless{}tastypie.fields.IntegerField object at 0xaa96b6ec\textgreater{}, `shape\_len': \textless{}tastypie.fields.FloatField object at 0xaa96b74c\textgreater{}, u'resource\_uri': \textless{}tastypie.fields.CharField object at 0xaa96b60c\textgreater{}\}}}
\end{fulllineitems}

\index{declared\_fields (map.api.CityBorderDetailResource attribute)}

\begin{fulllineitems}
\phantomsection\label{api/map:map.api.CityBorderDetailResource.declared_fields}\pysigline{\code{CityBorderDetailResource.}\bfcode{declared\_fields}\strong{ = \{\}}}
\end{fulllineitems}


\end{fulllineitems}

\index{CityBorderResource (class in map.api)}

\begin{fulllineitems}
\phantomsection\label{api/map:map.api.CityBorderResource}\pysiglinewithargsret{\strong{class }\code{map.api.}\bfcode{CityBorderResource}}{\emph{api\_name=None}}{}
Bases: \code{tastypie.resources.ModelResource}

API resource for the geographical information of the city border
\index{CityBorderResource.Meta (class in map.api)}

\begin{fulllineitems}
\phantomsection\label{api/map:map.api.CityBorderResource.Meta}\pysigline{\strong{class }\bfcode{Meta}}~\index{filtering (map.api.CityBorderResource.Meta attribute)}

\begin{fulllineitems}
\phantomsection\label{api/map:map.api.CityBorderResource.Meta.filtering}\pysigline{\bfcode{filtering}\strong{ = \{`name': `exact'\}}}
\end{fulllineitems}

\index{object\_class (map.api.CityBorderResource.Meta attribute)}

\begin{fulllineitems}
\phantomsection\label{api/map:map.api.CityBorderResource.Meta.object_class}\pysigline{\bfcode{object\_class}}
alias of \code{CityBorder}

\end{fulllineitems}

\index{queryset (map.api.CityBorderResource.Meta attribute)}

\begin{fulllineitems}
\phantomsection\label{api/map:map.api.CityBorderResource.Meta.queryset}\pysigline{\bfcode{queryset}\strong{ = {[}\textless{}CityBorder: Chicago\textgreater{}{]}}}
\end{fulllineitems}

\index{resource\_name (map.api.CityBorderResource.Meta attribute)}

\begin{fulllineitems}
\phantomsection\label{api/map:map.api.CityBorderResource.Meta.resource_name}\pysigline{\bfcode{resource\_name}\strong{ = `cityborder'}}
\end{fulllineitems}


\end{fulllineitems}

\index{base\_fields (map.api.CityBorderResource attribute)}

\begin{fulllineitems}
\phantomsection\label{api/map:map.api.CityBorderResource.base_fields}\pysigline{\code{CityBorderResource.}\bfcode{base\_fields}\strong{ = \{`box': \textless{}tastypie.fields.DictField object at 0xaa96b7ec\textgreater{}, `shape\_area': \textless{}tastypie.fields.FloatField object at 0xaa96bb2c\textgreater{}, `center': \textless{}tastypie.fields.CharField object at 0xaa96b7ac\textgreater{}, `objectid': \textless{}tastypie.fields.IntegerField object at 0xaa96bacc\textgreater{}, `geojson': \textless{}tastypie.fields.CharField object at 0xaa96b6ac\textgreater{}, `name': \textless{}tastypie.fields.CharField object at 0xaa96bb0c\textgreater{}, `geom': \textless{}tastypie.fields.CharField object at 0xaa96bbac\textgreater{}, `shape\_len': \textless{}tastypie.fields.FloatField object at 0xaa96bb6c\textgreater{}, u'id': \textless{}tastypie.fields.IntegerField object at 0xaa96baec\textgreater{}, u'resource\_uri': \textless{}tastypie.fields.CharField object at 0xaa96ba0c\textgreater{}\}}}
\end{fulllineitems}

\index{declared\_fields (map.api.CityBorderResource attribute)}

\begin{fulllineitems}
\phantomsection\label{api/map:map.api.CityBorderResource.declared_fields}\pysigline{\code{CityBorderResource.}\bfcode{declared\_fields}\strong{ = \{`box': \textless{}tastypie.fields.DictField object at 0xaa96b7ec\textgreater{}, `center': \textless{}tastypie.fields.CharField object at 0xaa96b7ac\textgreater{}, `geojson': \textless{}tastypie.fields.CharField object at 0xaa96b6ac\textgreater{}\}}}
\end{fulllineitems}


\end{fulllineitems}



\subsection{map.load module}
\label{api/map:module-map.load}\label{api/map:map-load-module}\index{map.load (module)}\index{run() (in module map.load)}

\begin{fulllineitems}
\phantomsection\label{api/map:map.load.run}\pysiglinewithargsret{\code{map.load.}\bfcode{run}}{\emph{verbose=True}}{}
load the shape files to the database
\begin{quote}\begin{description}
\item[{Parameters}] \leavevmode
\textbf{\texttt{verbose}} -- control message outputs of the method

\end{description}\end{quote}

\end{fulllineitems}



\subsection{map.models module}
\label{api/map:module-map.models}\label{api/map:map-models-module}\index{map.models (module)}\index{CityBorder (class in map.models)}

\begin{fulllineitems}
\phantomsection\label{api/map:map.models.CityBorder}\pysiglinewithargsret{\strong{class }\code{map.models.}\bfcode{CityBorder}}{\emph{*args}, \emph{**kwargs}}{}
Bases: \code{django.db.models.base.Model}

A model holding the values and attributes of a city border or map
\index{CityBorder.DoesNotExist}

\begin{fulllineitems}
\phantomsection\label{api/map:map.models.CityBorder.DoesNotExist}\pysigline{\strong{exception }\bfcode{DoesNotExist}}
Bases: \code{django.core.exceptions.ObjectDoesNotExist}

\end{fulllineitems}

\index{CityBorder.MultipleObjectsReturned}

\begin{fulllineitems}
\phantomsection\label{api/map:map.models.CityBorder.MultipleObjectsReturned}\pysigline{\strong{exception }\code{CityBorder.}\bfcode{MultipleObjectsReturned}}
Bases: \code{django.core.exceptions.MultipleObjectsReturned}

\end{fulllineitems}

\index{box (map.models.CityBorder attribute)}

\begin{fulllineitems}
\phantomsection\label{api/map:map.models.CityBorder.box}\pysigline{\code{CityBorder.}\bfcode{box}}
get box or envelope of the city border
\begin{quote}\begin{description}
\item[{Return type}] \leavevmode
vertices of the box or envelope of the city border

\end{description}\end{quote}

\end{fulllineitems}

\index{center (map.models.CityBorder attribute)}

\begin{fulllineitems}
\phantomsection\label{api/map:map.models.CityBorder.center}\pysigline{\code{CityBorder.}\bfcode{center}}
get center point of the gemotry
\begin{quote}\begin{description}
\item[{Return type}] \leavevmode
center point of the geomtry in geojson format

\end{description}\end{quote}

\end{fulllineitems}

\index{generateGrid() (map.models.CityBorder method)}

\begin{fulllineitems}
\phantomsection\label{api/map:map.models.CityBorder.generateGrid}\pysiglinewithargsret{\code{CityBorder.}\bfcode{generateGrid}}{\emph{size}}{}
generate a grid overlaying the city
\begin{quote}\begin{description}
\item[{Parameters}] \leavevmode
\textbf{\texttt{size}} -- dimension of the cell for the grid

\item[{Return type}] \leavevmode
an array of cells that forms the grid

\end{description}\end{quote}

\end{fulllineitems}

\index{geojson (map.models.CityBorder attribute)}

\begin{fulllineitems}
\phantomsection\label{api/map:map.models.CityBorder.geojson}\pysigline{\code{CityBorder.}\bfcode{geojson}}
generate geojson of the CityBorder
:rtype: city border's geojson

\end{fulllineitems}

\index{geom (map.models.CityBorder attribute)}

\begin{fulllineitems}
\phantomsection\label{api/map:map.models.CityBorder.geom}\pysigline{\code{CityBorder.}\bfcode{geom}}
\end{fulllineitems}

\index{objects (map.models.CityBorder attribute)}

\begin{fulllineitems}
\phantomsection\label{api/map:map.models.CityBorder.objects}\pysigline{\code{CityBorder.}\bfcode{objects}\strong{ = \textless{}django.db.models.manager.Manager object\textgreater{}}}
\end{fulllineitems}


\end{fulllineitems}



\subsection{map.utils module}
\label{api/map:module-map.utils}\label{api/map:map-utils-module}\index{map.utils (module)}
Utility methods to generate GeoJSON files and compute distances
\index{generateGeoJson() (in module map.utils)}

\begin{fulllineitems}
\phantomsection\label{api/map:map.utils.generateGeoJson}\pysiglinewithargsret{\code{map.utils.}\bfcode{generateGeoJson}}{\emph{geometry}}{}
generate GeoJSON from a geomtry for easy plotting of the borders
\begin{quote}\begin{description}
\item[{Parameters}] \leavevmode
\textbf{\texttt{geometry}} -- the generated GeoJSON of a geometry

\item[{Rtpye}] \leavevmode
GeoJson format of geometry

\end{description}\end{quote}

\end{fulllineitems}

\index{getNextPoint() (in module map.utils)}

\begin{fulllineitems}
\phantomsection\label{api/map:map.utils.getNextPoint}\pysiglinewithargsret{\code{map.utils.}\bfcode{getNextPoint}}{\emph{point}, \emph{offset}, \emph{bearing}}{}
get next point from the given point at a distance and bearing
\begin{quote}\begin{description}
\item[{Parameters}] \leavevmode\begin{itemize}
\item {} 
\textbf{\texttt{point}} -- point to be based upon the next point

\item {} 
\textbf{\texttt{offset}} -- the distance to be calculated to get the next point

\item {} 
\textbf{\texttt{bearing}} -- the direction or angle to where the next point will be calculated

\end{itemize}

\item[{Return type}] \leavevmode
GEOS Point representing the longitude and latitude of the new point

\end{description}\end{quote}

\end{fulllineitems}



\subsection{map.views module}
\label{api/map:map-views-module}\label{api/map:module-map.views}\index{map.views (module)}\index{FetchGridView (class in map.views)}

\begin{fulllineitems}
\phantomsection\label{api/map:map.views.FetchGridView}\pysiglinewithargsret{\strong{class }\code{map.views.}\bfcode{FetchGridView}}{\emph{**kwargs}}{}
Bases: \code{django.views.generic.base.View}

Fetch grid of a map
\index{get() (map.views.FetchGridView method)}

\begin{fulllineitems}
\phantomsection\label{api/map:map.views.FetchGridView.get}\pysiglinewithargsret{\bfcode{get}}{\emph{request}, \emph{*args}, \emph{**kwargs}}{}
get the grid of the map based on supplied pk, size
\begin{quote}\begin{description}
\item[{Return type}] \leavevmode
HTTPResponse with the dumped geojson

\end{description}\end{quote}

\end{fulllineitems}


\end{fulllineitems}



\chapter{Indices and tables}
\label{index:indices-and-tables}\begin{itemize}
\item {} 
\DUrole{xref,std,std-ref}{genindex}

\item {} 
\DUrole{xref,std,std-ref}{modindex}

\item {} 
\DUrole{xref,std,std-ref}{search}

\end{itemize}


\renewcommand{\indexname}{Python Module Index}
\begin{theindex}
\def\bigletter#1{{\Large\sffamily#1}\nopagebreak\vspace{1mm}}
\bigletter{c}
\item {\texttt{core.finders.mustache\_template}}, \pageref{api/core.finders:module-core.finders.mustache_template}
\item {\texttt{crime.admin}}, \pageref{api/crime:module-crime.admin}
\item {\texttt{crime.api}}, \pageref{api/crime:module-crime.api}
\item {\texttt{crime.models}}, \pageref{api/crime:module-crime.models}
\item {\texttt{crime.views}}, \pageref{api/crime:module-crime.views}
\item {\texttt{crimeprediction.error\_analysis}}, \pageref{api/crimeprediction:module-crimeprediction.error_analysis}
\item {\texttt{crimeprediction.experiments}}, \pageref{api/crimeprediction:module-crimeprediction.experiments}
\item {\texttt{crimeprediction.network}}, \pageref{api/crimeprediction:module-crimeprediction.network}
\item {\texttt{crimeprediction.vectorize}}, \pageref{api/crimeprediction:module-crimeprediction.vectorize}
\item {\texttt{crimeprediction.views}}, \pageref{api/crimeprediction:module-crimeprediction.views}
\indexspace
\bigletter{m}
\item {\texttt{map.api}}, \pageref{api/map:module-map.api}
\item {\texttt{map.load}}, \pageref{api/map:module-map.load}
\item {\texttt{map.models}}, \pageref{api/map:module-map.models}
\item {\texttt{map.utils}}, \pageref{api/map:module-map.utils}
\item {\texttt{map.views}}, \pageref{api/map:module-map.views}
\end{theindex}

\renewcommand{\indexname}{Index}
\printindex
\end{document}
