%*******************************************************************************
%*********************************** Second Chapter *****************************
%*******************************************************************************

\chapter{Review of Related Literature}  %Title of the Second Chapter

%********************************** %First Section  **************************************
\section{Crime} %Section - 2.1
The formal definition of crime varies widely among criminologists but Edwin Sutherland’s short one may be sufficient. According to him, crime is “[a]n unlawful act is not defined as criminal by the fact that it is punished, but by the fact that it is punishable.” (as cited in Brown et. al., 2010) But what is punishable may be relative to what the crime is and where it happened. As Sutherland continued, an act can be considered crime in its essentiality once the State as which the jurisdiction of the act prevails deems such act as an injury and punishable by law.
In the Philippines, as per the Philippine National Police, crimes can be classified into two: index crimes and non-index crimes (as cited in Tumulak, 2015). Index crimes are crimes that involves victims such as murder, homicide, physical injury and rape; or against property such as robbery, theft, and burglary. On the other hand, non-index crimes are violations of special laws such as illegal logging or local ordinances. In the Philippines, these classifications are used for statistical purposes and to create a standardized definition of crime classification. (Tumulak, 2015)
