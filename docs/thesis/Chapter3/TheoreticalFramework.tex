%*******************************************************************************
%*********************************** Third Chapter *****************************
%*******************************************************************************

\chapter{Study Frameworks}  %Title of the Third Chapter

\section{Theoretical Framework}
    The theoretical framework presents the theories/concepts that govern the flow of the study. The first concept of this study is grid mapping. Grid mapping avoids problems of dealing with different spatial measurements and helps in quick identification of spatial patterns. After mapping the grid would be generating the relevant data. Data must be properly vectorized, which means it is in meaningful representation and is preprocessed for convenient handling. Model training is the learning of patterns and rules in a given dataset. This is essential in order to achieve the objective of this study, which is to accurately predict crime hotspots. In order to evaluate the accuracy of the model, experiments must be performed and measured. This makes the whole study credible and valid.


\section{Operational Framework}
    The operation framework illustrates the flow of processes followed by the study. The first step is generating the grid. The grid is generated using geospatial methods provided by GeoDjango and GeoPy. Next, data is collected and preprocessed into a vector. Instances of the CriminalRecord object are fetched and filtered by period (weekly, monthly, yearly) and type. The presence and absence of crime in a cell is represented by 1 or -1. A row of data consists of -1s and 1s for all the grid, generated by period. The data generated is then used as input in training the model. The model is build on top of a library called Keras, using a recurrent neural network with LSTM layer. The model is then trained with the dataset, seperated to training set and set test by 70\% and 30\% respectively. After the model is trained, experiments are conducted while measuring the performance of the model through various metrics.