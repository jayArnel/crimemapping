\begin{abstract}
Crime hotspots prediction is helpful for the police to determine the areas where crime will most likely to occur. Concentrating on these areas will help prevent crime and manage resources efficiently. One useful characteristic of crime is that it is recurrent in nature. By taking this pattern into account, the time an area becomes a hotspot can be predicted. To detect this pattern, this study uses a Recurrent Neural Network (RNN) with Long Short-term Memory Architecture (LSTM) to predict crime hotspots over time. The study area is Chicago City, IL, USA. The type of hotspot used is grid thematic mapping, where a grid of uniform cell of a certain dimensions is overlaying the study area. The LSTM network learns through a data of the presence or absence of crime in each cell for different timestep or period of time. The study experiments on different cell dimensions (500mx500m. 750mx750m, 1000mx1000m) and timesteps (weekly, monthly and yearly.) Additionally, to support recurrence, this study experiments on seasonal and non-seasonal crime data. The network performed better on 750mx750m cell dimension, monthly timestep and seasonal data. Error analysis show that the model can further be improved by using more features or more training data.
\end{abstract}
