% ************************** Thesis Abstract *****************************
% Use `abstract' as an option in the document class to print only the titlepage and the abstract.
\begin{abstract}
\addcontentsline{toc}{chapter}{Abstract}

\begin{spacing}{1}
Crimes have negative impact to people and society. Hence, it is important to prevent criminal activities to happen or at least deal with them immediately when they occur. One way to help prevent and deal with criminal activities is to understand crime hotspots. Predicting and mapping crime hotspots will be a significant help to the police. It will help them to efficiently manage their assets while taking watch in areas where criminal activities are more likely to occur. This study aims to develop a model that will accurately predict crime hotspots. The study used a machine learning model using a Recurrent Neural Network (RNN) with Long Short-term Memory (LSTM) architecture to take advantage of the recurrent nature of crime. The dataset used is the Chicago’s online dataset of criminal records from 2001-present. The model is trained to map crime hotspots by predicting the presense or absence of crime in different cells in a grid overlaying the city of Chicago. The model will learn through a sequence of "grid snapshots" over time, which is a record of the presence or absence of crime, represented by 1s and -1s respectively, by a given timestep. The model will be trained and experimented under different values for the cell dimension of the grid and timestep of the grid snapshots. The different cell dimensions are 500mx500m, 750mx750m and 1000mx1000m. The different timesteps are weekly, monthly and yearly. In addition, the model will be experimented with seasonal and non-seasonal data. The results showed that the model trained better using the cell dimension 750mx750m, monthly timestep and seasonal data. Error analysis was applied to the model to evaluate its performance if it suffered high variance or high bias. The training error and cross-validation error of the model is plotted for different training sample sizes. The analysis shows that the model performs better on monthly timestep because the training size for the model in yearly timestep is so small that the model overfits in training and the training size for the model in weekly timestep, on the other hand, is too large that the model underfits. The cell dimension of the model shows no effect to the variance or bias of the model. Hence, the model should be flexible enough for the different parameter values it can have. If the end-user of user wishes to predict crime hotspots weekly, or even daily, additional features must be introduced. The yearly timestep can best be used when the data spans enough years for the model to avoid overfitting.

\end{spacing}
\end{abstract}
