%********************************** %First Section  **************************************
\section{Rationale} %Section - 1.1 

    Crimes have always been a detrimental problem to humanity. Although the world has seen a general drop in crime trends especially in developed countries (van Dijk, van Kesteren, & Smit, 2007), it is still undeniable that crimes have negative impact on the quality of life. Victims of crimes suffer receded productivity in performing their roles in society such as parenting, intimate relationships and occupational responsibilities (Hanson, Genelle, Begle, & Hubel, 2010). Furthermore, it is not just the direct victims of crime that suffer but those surrounding them as well. In households with intimate-partner violence (IPV), for instance, children experience the consequences of maltreatment to one of their parents (Casanueva, Martin, Runyan, & Bradley, 2008). Being in a violent household gives them psychological and emotional stress that may affect their character development. In addition, crime doesn’t only have social implications but have economic effects as well. As many as 30\% of failed businesses can be attributed to criminal activities (Bressler, 2009). With all these adverse repercussions that crimes pose to society, it is of utmost importance that criminal activities be prevented.

    Unfortunately, crime prevention has been met with a lot of challenges. Police use traditional techniques to prepare for a possible criminal activity but in order to make effective prevention measures and precautions, the police must take advantage of criminological theories, crime analysis and technology (Polat, 2007).

    Crime analysis aims in researching and utilizing systematic methods and data to support police and criminologists in dealing with crime, as well as to provide information to stakeholders (Santos, 2013). One important aspect of crime analysis is crime mapping which is dedicated to locating high-crime areas called hotspots. Understanding and mapping crime hotspots will be very helpful to police officials in determining which areas within their jurisdiction can a criminal activity be more likely to happen (Eck, 2005).

    This study aims to aid police officers by developing a machine learning model that will generate predictions of crime hotspots. The generated prediction will valuably help the police efficiently deploy their assets by directing their efforts to where it may be needed the most. The machine learning model for this study will be a Recurrent Neural Network (RNN) with a Long Short-Term Memory (LSTM) architecture. This kind of model aims to be more accurate in predicting and modelling criminal activities compared to other models already used before.