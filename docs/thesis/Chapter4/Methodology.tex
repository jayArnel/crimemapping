%*******************************************************************************
%*********************************** Fourth Chapter *****************************
%*******************************************************************************

\chapter{Methodology}  %Title of the Fourth Chapter


%********************************** %First Section  **************************************
\section{Application Development} %Section - 4.1
    
    The application was built using Django, a Python-based web framework.


%********************************** %Second Section  **************************************
\section{Area Boundaries and Grid Layout} %Section - 4.2

    The boundaries for the study area, Chicago, was accessed through MapTechnica.com. Since the boundaries is not easily accessible, the page’s JavaScript file was edited and injected with lines in order to get the boundaries of Chicago City. The boundaries are represented in GeoJSON file so that it can also be loaded to the Google Maps. The whole city is divided into grids of fixed sizes, each representing an area where prediction of an occurrence of criminal activity will be based upon.


%********************************** %Third Section  **************************************
\section{Dataset Representation} %Section - 4.3

    The data fetched from the dataset of Chicago’s criminal records is stored in a database where each row  contains the date and coordinates of each crime. Further processing of the data will create a vector array where each row contains a series of -1s and 1s for each grid in the map representing the absence and presence, respectively, of criminal activity.

    The dataset will be divided into three partitions – training, validation and testing.